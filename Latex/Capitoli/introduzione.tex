\lvli{Introduzione}
\lvlii{Legge di Moore}
È messa male.
\lvlii{Computer quantistico/adiabatico}
Da circa vent'anni a questa parte è crescito l'interesse per i computer quantistici con la speranza che possano fornire prestazioni superiori rispetto ai computer classici grazie all'esistenza delle super position nella meccanica quantiistica rispetto quella classica. Shor ha dimostrato che utilizzando algoritmi quantistici è possibile risolvere la fattorizzazione e l'estrazione dei logaritmi discreti in tempi polinomiali rispetto ai tempi superpolinomiali degli algoritmi classici. Successivamente Feynman introduce l'idea che questi algoritmi quantistici non possano essere simulati in computer classici visto che richiederebbero risorse esponenziali, da qui nasce l'esigenza della ricerca dei computer quantisti, ovvero computer in grado di utilizzare algoritmi quantistici con risorse polinomiali.
