\subsection{Postulati}
Nel campo scentifico esistono due tipi di postulati. Il primo tipo si basa sull'osservazione diretta ad esempio <<il calore fluisce spontaneamente dai corpi più caldi a quelli più freddi>>. Il secondo tipo è quello non immediato, per eprimerlo bisogna citare una serie di contenuti consecutivi ad esempio <<l'entropia dei sistemi isolati aumenta nel corso delle trasformazioni spontanee>>. I postulati della meccanica quantistica appartengono al secondo tipo.

\subsubsection{Stati e funzioni d'onda}
Il primo postulato esprime l'informazione per descrivere lo stato di un sistema.
\begin{quote}
\myindex{Postulato 1}: Lo stato del sistema è completamente descritto dalla funzione d'onda $\Psi(x_1, x_2, ..., t)$.
\end{quote}
In questo enunciato $x_1, x_2, ...$ sono le coordinate spaziali, $t$ è il tempo e $\Psi$ è la \myindex{funzione d'onda}. Da questo postulato si capisce che conoscere la funzione d'onda significa conoscere lo stato del sistema. La funzione d'onda $\psi_{a, b, ...}$ può essere specificata da un insieme di <<marcatori>> chiamati \myindex{numeri quantici} $a, b, ...$. Lo stato può dunque essere completamente descritto anche enumerando i valori dei numeri quantici che lo definiscono.

\subsubsection{La prescrizione fondamentale}
Il secondo postulato riguarda la scelta degli operatori.
\begin{quote}
  \myindex{Postulato 2}: Le osservabili sono rappresentate da operatori scelti in modo da soddisfare le relazioni di commutazione:
  $$ [\hat{q}, \hat{p_{q'}}] = i \hbar \delta_{qq'} \qquad [\hat{q}, \hat{q}'] = 0 \qquad [\hat{p_q}, \hat{p_{q'}}] = 0$$
\end{quote}
In questo enunciato $\hat{q}, \hat{q}'$ rappresentano delle coordinate $x, y, z$, $\hat{p_q}, \hat{p_{q'}}$ sono i rispettivi momenti lineari. La prescrizione fondamentale è un postulato che non può essere né provato né dedotto. Offre una base per la scelta degli operatori che devono rispettare le relazioni di commutazione.

\subsubsection{Il risultato delle misure}
Il terzo postulato introduce il collegamento tra funzione d'onda ed operatori. Stabilisce inoltre il legame fra i calcoli formali e quelli sperimentali.
\begin{quote}
\myindex{Postulato 3}: Quando un sistema è descritto dalla funzione d'onda $\psi$, il valor medio dell'osservazione $\omega$, in una serie di misure, è uguale al valore atteso dell'operatore corrispondente $\hat{\Omega}$.
\end{quote}
Data una funzione d'onda $\psi$ ed un opertore $\hat{\Omega}$ il \myindex{valore atteso} di un operatore è definito:
$$\hat{\left \langle \Omega \right \rangle} = \frac{ \int \psi^{\star} \hat{\Omega} \psi \,d\tau }{ \int \psi^{\star} \psi \,d\tau }$$
Scegliendo una \myindex{funzione d'onda normalizzata} ovvero una funzione che rispetti questa proprietà:
$$\int \psi^{\star} \psi \,d\tau = 1$$
il valore atteso di questa funzione risulta:
$$\hat{\left \langle \Omega \right \rangle} = \int \psi^{\star} \hat{\Omega} \psi \,d\tau$$
Per spiegare il postulato 3 prendiamo $\psi$ normalizzata in modo che sia un'autofunzione di $\hat{\Omega}$ ed abbia autovalore $\omega$.
$$\hat{\left \langle \Omega \right \rangle} = \int \psi^{\star} \hat{\Omega} \psi \,d\tau = \int \psi^{\star} \omega \psi \,d\tau = \omega \int \psi^{\star} \psi \,d\tau = \omega$$
In questo caso $\psi$ rappresenta il sistema sul quale andiamo ad effettuare le misure, $\omega$ rappresenta il valor medio delle misure effettuate ed $\hat{\Omega}$ rappresenta l'operatore corrispondente.
Supponiamo invece che $\psi$ sia un autostato dell'hamiltoniano e non sia un autostato di $\hat{\Omega}$. In questo caso $\psi$ si potrà scrivere come combinazione lineare di funzioni con autostato di $\hat{\Omega}$.
$$\hat{\Omega} = \sum_n c_n \psi_n \qquad ed \qquad \hat{\Omega} \psi_n = \omega_n \psi_n$$
Il valore di attesa risulterà:
$$\hat{\left \langle \Omega \right \rangle} = \int \psi^{\star} \hat{\Omega} \psi \,d\tau = \int {\left(\sum_m c_m \psi_m\right)}^{\star} \hat{\Omega} \sum_n c_n \psi_n \,d\tau = \sum_{m,n} c_m^{\star} c_n \int \psi_m^{\star} \hat{\Omega} \psi_n \, d\tau$$
$$\hat{\left \langle \Omega \right \rangle} = \sum_{m,n} c_m^{\star} c_n \omega_n \int \psi_m^{\star} \psi_n \, d\tau$$
È possibile compattare ancora questa funzione considerando che le autofunzioni formano un insieme ortonormale e quindi il prodotto $\psi_m^{\star} \psi_n$ si annulla sempre tranne nel caso in cui $n = m$. Dalla funzione precedente si ottiene dunque:
$$\hat{\left \langle \Omega \right \rangle} = \sum_{n} c_n^{\star} c_n \omega_n \int \psi_n^{\star} \psi_n \, d\tau = \sum_n {\left| c_n \right|}^2 \omega_n$$
Da queste ultime considerazioni si può ricavare una versione più dettagliata del postulato 3.
\begin{quote}
  \myindex{Postulato 3'}: Quando $\psi$ è un'autofunzione dell'operatore $\hat{\Omega}$ la determinazione della proprietà $\hat{\Omega}$ fornisce sempre un unico risultato: l'autovalore corrispondente $\omega$. Quando $\psi$ non è un'autofunzione di $\hat{\Omega}$, una misura della proprietà fornisce un unico risultato, che coincide con uno degli autovalori di $\hat{\Omega}$, e la probabilità che si sia misurato un particolare autovalore $n$ è uguale a ${\left|c_n\right|}^2$, dove $c_n$ è il coefficente dell'autofunzione $\psi_n$ nello sviluppo della funzione d'onda $\psi$.
\end{quote}

\subsubsection{L'interpretazione della funzione d'onda}
Spesso chiamato \myindex{interpretazione di Born} il quarto postulato riguarda l'interpretazione della funzione d'onda.
\begin{quote}
  \myindex{Postulato 4}: La probabilità di rinvenire una particella nell'elemento di volume $d\tau$ al punto $r$ è proporzionale a ${\left|\psi(r)\right|}^2 d\psi$.
\end{quote}
Si chiamata \myindex{densità di probabilità} la funzione ${\left|\psi(r)\right|}^2 d\psi$ perché moltiplicata per un certo volume, essa fornisce la probabilità di trovare la particella in quella regione. Nel caso monodimensionale:
$$\mathbb{P}(a, b) = \int_a^b {\left|\psi(x)\right|}^2 \, dx$$
La funzione $\psi$ viene anche detta \myindex{ampiezza di probabilità}, nell'esempio del caso monodimensionale si può notare il legame tra $\psi$ e l'ampiezza della funzione $\mathbb{P}$:
$$\mathbb{P}(x) = {\left|\psi(x)\right|}^2$$
L'interpretazione di Born, in altri termini, vincola la funzione ad essere integrabile al quadrato:
$$\int {\left|\psi\right|}^2 \, d\tau < \infty$$
Questa formula indica che la probabilità di trovare una particella in una regione dello spazio è finita e vale $1$ per le funzioni normalizzate. Da qui si deduce facilmente che $\psi \to 0$ per $ x \to \pm \infty$ altrimenti l'integrale sarebbe infinito.

\subsubsection{L'equazione della funzione d'onda}
Il quinto ed ultimo postulato, anche noto come \myindex{equazione di Schr\"odinger}, espone come la funzione d'onda evolve nel tempo.
\begin{quote}
  \myindex{Postulato 5}: La funzione d'onda $\Psi(r_1, r_2, ..., t)$ evolve nel tempo secondo l'equazione
  $$i\hbar \frac{\partial \Psi}{\partial t} = \hat{H} \Psi$$
\end{quote}
L'operatore $\hat{H}$ rappresenta l'operatore di energia chiamato hamiltoniano. Nel caso di una particella libera si può sostituire con la seguente formula:
$$i \hbar \frac{\partial \Psi}{\partial t} = - \frac{\hbar^2}{2 m} \frac{\partial^2 \Psi}{\partial x^2} + V(x) \Psi$$
Quando l'energia potenziale è indipendente dal tempo, essa si può separare in funzioni legate al tempo e funzioni legate allo spazio. Applicando la tecnica della \myindex{separazione delle variabili} possiamo riscrivere $\Psi$:
$$\Psi(x, t) = \psi(x) \theta(t)$$
Mettendo al primo membro solo i termini che dipendono da $t$ e al secondo solo quelli che dipendono da $x$
$$i \hbar \frac{1}{\theta(t)} \frac{\partial \theta(t)}{\partial t} = - \frac{\hbar^2}{2 m} \frac{1}{\psi(x)} \frac{\partial^2 \psi(x)}{\partial x^2} + V(x)$$
si nota che se $x$ cambia, il secondo membro rimane costante, di conseguenza essendo il primo membro uguale al secondo anche il primo rimane costante. Chiameremo questa costante $E$ in modo che ricordi le dimensioni dell'energia di cui essa ne fa parte. Ne conseguono due funzioni:
$$E \theta(t) = i \hbar \frac{1}{\theta(t)} \frac{\partial \theta(t)}{\partial t}$$
$$E \psi(x) = - \frac{\hbar^2}{2 m} \frac{1}{\psi(x)} \frac{\partial^2 \psi(x)}{\partial x^2} + V(x)$$
La prima di queste due equazioni ammette come soluzione
$$\theta \propto e^{\frac{-i E t}{\hbar}}$$
e dunque la funzione completa avra come soluzione
$$\Psi(x, t) = \psi(x) e^{\frac{-i E t}{\hbar}}$$
più semplicemente potremo scrivere per $\psi(x)$ l'equazione:
$$\hat{H} \psi(x) = E \psi(x)$$
Quest'ultima equazione prende il nome di equazione di Scr\"odinger indipendente dal tempo.
Quando l'energia potenziale è indipendente dal tempo, per descrivere il moto che dipende dal tempo, basterà prendere la funzione $\psi(x)$ e moltiplicarla per la modulazione di fase $e^{\frac{-i E t}{\hbar}}$. Nonostante la funzione d'onda oscilli nel tempo la quantità ${\left|\Psi\right|}^2$ rimane costate, in uno \myindex{stato stazionario}:
$$\Psi^{\star} \Psi = {\left(\psi(x) e^{\frac{-i E t}{\hbar}}\right)}^{\star} \left(\psi(x) e^{\frac{-i E t}{\hbar}}\right) = \psi^{\star} \psi$$
