\subsection{Postulati}
Nel campo scentifico esistono due tipi di postulati. Il primo tipo si basa sull'osservazione diretta ad esempio <<il calore fluisce spontaneamente dai corpi più caldi a quelli più freddi>>. Il secondo tipo è quello non immediato, per eprimerlo bisogna citare una serie di contenuti consecutivi ad esempio <<l'entropia dei sistemi isolati aumenta nel corso delle trasformazioni spontanee>>. I postulati della meccanica quantistica appartengono al secondo tipo.

\subsubsection{Stati e funzioni d'onda}
Il primo postulato esprime l'informazione sufficente e necessaria per descrivere lo stato di un sistema.
\begin{quote}
\textbf{Postulato 1}: Lo stato del sistema è completamente descritto dalla funzione d'onda $\Psi(x_1, x_2, ..., t)$.
\end{quote}
In questo enunciato $x_1, x_2, ...$ sono le coordinate spaziali, $t$ è il tempo e $\Psi$ è la \textbf{funzione d'onda}. Da questo postulato si capisce che conoscere la funzione d'onda significa conoscere lo stato del sistema. La funzione d'onda $\psi_{a, b, ...}$ può essere specificata da un insieme di <<marcatori>> chiamati \textbf{numeri quantici} $a, b, ...$. Lo stato può dunque essere completamente descritto anche enumerando i valori dei numeri quantici che lo definiscono.

\subsubsection{La prescrizione fondamentale}
Il secondo postulato riguarda la scelta degli operatori.
\begin{quote}
  \textbf{Postulato 2}: Le osservabili sono rappresentate da operatori scelti in modo da soddisfare le relazioni di commutazione:
  $$ [\hat{q}, \hat{p_{q'}}] = i \hbar \delta_{qq'} \qquad [\hat{q}, \hat{q}'] = 0 \qquad [\hat{p_q}, \hat{p_{q'}}] = 0$$
\end{quote}
In questo enunciato $\hat{q}, \hat{q}'$ rappresentano delle coordinate $x, y, z$, $\hat{p_q}, \hat{p_{q'}}$ sono i rispettivi momenti lineari.
% \subsubsection{Il risultato delle misure}
% \subsubsection{L'interpretazione della funzione d'onda}
% \subsubsection{L'equazione della funzione d'onda}
