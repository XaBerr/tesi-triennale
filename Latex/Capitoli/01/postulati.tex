\subsection{Postulati}
Nel campo scentifico esistono due tipi di postulati. Il primo tipo si basa sull'osservazione diretta ad esempio <<il calore fluisce spontaneamente dai corpi più caldi a quelli più freddi>>. Il secondo tipo è quello non immediato, per eprimerlo bisogna citare una serie di contenuti consecutivi ad esempio <<l'entropia dei sistemi isolati aumenta nel corso delle trasformazioni spontanee>>. I postulati della meccanica quantistica appartengono al secondo tipo.

\subsubsection{Stati e funzioni d'onda}

% \subsubsection{La prescrizione fondamentale}
% \subsubsection{Il risultato delle misure}
% \subsubsection{L'interpretazione della funzione d'onda}
% \subsubsection{L'equazione della funzione d'onda}
