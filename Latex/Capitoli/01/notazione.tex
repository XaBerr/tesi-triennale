\lvlii{Notazione di Dirac}
\lvliii{Braket}
Nella notazione braket ci sono due simboli che ricordano l'utilizzo delle parentesi il cui termine inglese è appunto brackets. Il primo simbolo si chiama \idx{bra} $\langle m |$ ed indica il complesso coniugato della funzione $\psi_m$.
$$\langle m | = \psi_m^{\star}$$
Il secondo simbolo si chiama \idx{ket} $| n \rangle$ ed indica la funzione stessa $\psi_n$.
$$| n \rangle = \psi_n$$
Accoppiando i due simboli $\langle m | n \rangle$ si indica l'integrale:
$$\langle m | n \rangle = \int \psi_m^{\star} \psi_n\, d\tau$$
In particolare l'integrale di normalizzazione avrà la seguente scrittura:
$$\langle n | n \rangle = \int \psi_n^{\star} \psi_n\, d\tau = 1$$
Data una famiglia di funzioni $\psi_n$ la \idx{condizione di ortonormalità} è rispettata se vale la seguente proprietà
$$\langle m | n \rangle = \int \psi_m^{\star} \psi_n\, d\tau = \delta_{mn}$$
dove $\delta_{mn}$ si chiama \idx{delta di Kronecker} e vale $1$ se $m = n$ mentre vale $0$ nel resto dei casi.
Questa notazione supporta anche l'utilizzo degli operatori, dato l'opertore $\hat{\Omega}$ abbiamo che:
$$\langle m | \hat{\Omega} | n \rangle = \int \psi_m^{\star} \hat{\Omega} \psi_n\, d\tau$$
Facciamo notare che il caso $\langle m | \hat{\Omega} | n \rangle$ e $\langle m | n \rangle$ sono uguali sottointendendo che nel secondo caso l'operatore è $\hat{1}$ che corrisponde alla moltiplicazione per $1$.
$$\langle m | \hat{1} | n \rangle = \langle m | n \rangle$$
È anche facile dimostrare che vale la seguente proprietà:
$$\langle m | n \rangle = {\langle n | m \rangle}^{\star}$$
