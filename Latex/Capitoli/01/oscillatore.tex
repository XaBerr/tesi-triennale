\subsection{Oscillatore armonico}
\subsubsection{Moto traslazionale}
Il moto più semplice da introdurre è quello di una particella che si muove in uno spazio monodimensionale a potenziale costante. L'hamiltoniana di questa particella è:
$$\hat{H} = - \frac{\hbar^2}{2m} \partial_x^2$$
L'equazione di Scr\"odinger associata indipendente del tempo $\hat{H}\psi = \hat{E}\psi$ ha come soluzione:
$$\psi = A e^{i k x} + B e^{- i k x} \qquad k = \sqrt{\frac{2 m E}{\hbar^2}}$$
I coefficenti $A, B$ sono indicativi della direzione di moto della particella, ad esempio se la particella viaggia da destra a sinistra avrà sicuramente il coefficente $A = 0$, invece se viaggia da sinistra a destra avrà il coefficente $B = 0$.
Dal coefficente $k$ possiamo invece notare che:
$$k\hbar = \frac{\hbar}{\hbar} \sqrt{2 m E \hbar} = \frac{\hbar}{\hbar} \sqrt{2 m \frac{p^2}{2 m}} = p \qquad con \qquad E = \frac{p^2}{2m}$$
Riscrivendo la funzine di $\psi$ come
$$\psi = C \cos{kx} + D \sin{kx}$$
ovvero come
$$\psi = C \cos{\frac{p x 2 \pi}{h}} + D \sin{\frac{p x 2 \pi}{h}}$$
ne consegue che la lunghezza d'onda è $\lambda = \frac{2\pi}{k}$ e
$$p = \frac{h}{\lambda}$$
Quest'ultima relazione prende il nome di \myindex{relazione di de Broglie}.

\subsubsection{Densità di flusso}
Data una funzione d'onda $\Psi$ si definisce densità di flusso:
$$J_x = \frac{\hbar}{2mi}\left(\Psi^{\star} \frac{\partial \Psi}{\partial x} - \Psi \frac{\partial \Psi^{\star}}{\partial x}  \right)$$

\subsubsection{Pacchetto d'onde}
Un pacchetto d'onde è uno stato di energia imprecisa descrivibile dalla sovrapposizione di energie precise.
La funzione d'onda risulterà di ampiezza non nulla in una piccola regione dello spazio e nulla in qualunque altro posto.
Avrà come forma
$$\Psi(x, t) = \int g(k) \Psi_k(x, t) \, dk \qquad con \qquad \Psi_k(x, t) = A e^{ikx} e^{\frac{-i E_k t}{\hbar}}$$
dove $g(k)$ prenderà il nome di \myindex{funzione peso} e $\Psi_k$ sarà la famiglia d'onde che si sovrappongono per formare $\Psi$.

\subsubsection{Barriera di potenziale}
Classicamente una particella può superare una barriera di potenziale solo se la sua energia iniziale è superiore a quella del potenziale, in caso contrario la particella verrebbe riflessa. In meccanica quantistica la faccenda è più complessa. Vediamo ora come si comporta la particella a seconda che lo spessore della barriera sia infinito o finito.

\myparagraph{Parete di potenziale infinitamente spessa}
Prendiamo una regione dello spazio e dividiamola in due, la parte sinistra $(x < 0)$ che chiameremo \textit{Zona I} decidiamo avere potenziale $0$, mentre la parte destra $(x > 0)$ la nomineremo \textit{Zona II} ed avrà potenziale $V$. L'hamiltoniana dello spazio sarà:
$$H=\begin{cases} - \frac{\hbar^2}{2m} \partial_x^2, & \mbox{se } x<0 \mbox{ Zona I} \\ - \frac{\hbar^2}{2m} \partial_x^2 + V, & \mbox{se } x>0 \mbox{ Zona II}
\end{cases}$$
Consideriamo ora una particella libera che si muove da sinistra a desta con energia iniziale $E$, descriviamo la sua funzione d'onda nelle due zone:
\begin{equation}
\begin{aligned}
  \quad \mbox{Zona I: }  &\psi = Ae^{ikx} + Be^{-ikx} \qquad & con \qquad & k\hbar = \sqrt{2mE}\\
  \quad \mbox{Zona II: } &\psi = A'e^{ik'x} + B'e^{-ik'x} \qquad & con \qquad & k'\hbar = \sqrt{2m(E-V)}
\end{aligned}
\end{equation}
Nella seconda equazione $E - V < 0$ e qunque $k'$ risulta un numero immaginario, possiamo riscriverlo come $k' = i \kappa$ con $\kappa \in \mathbb{R}$ ed ottenere l'equazione:
$$\psi = A'e^{-\kappa'x} + B'e^{\kappa'x} \qquad  con \qquad \kappa\hbar = \sqrt{2m(E-V)}$$
A questo punto, sapendo che la particella non può stare nella \textit{Zona II}, teniamo l'esponenziale che tende a $0$ ed eliminiamo l'altro imponendo $B' = 0$. Per quanto velocemente tenda a $0$ l'esponenziale, rimarrà comunque una piccola probabilità di rinvenire la particella all'interno della barriera, questo effetto prende il nome di \myindex{penetrazione} e la distanza massima in cui si può trovare la particella all'interno della barriera è $\frac{1}{\kappa}$ e si chiama \myindex{profondità di penetrazione}.

\myparagraph{Parete di potenziale di spessore finito}
Prendiamo una regione dello spazio, questa volta la dividiamo in tre, la parte sinistra $(x < 0)$ e la parte destra $(x > L)$ che chiameremo rispettivamente \textit{Zona I} e \textit{Zona III} decidiamo avere potenziale $0$, mentre la parte centrale $(0 < x < L)$ la nomineremo \textit{Zona II} ed avrà potenziale $V$ dove $L$ è una lunghezza arbitraria finita. L'hamiltoniana dello spazio sarà leggermente diversa:
$$H=\begin{cases} - \frac{\hbar^2}{2m} \partial_x^2, & \mbox{se } x<0 \mbox{ Zona I} \\
 - \frac{\hbar^2}{2m} \partial_x^2 + V, & \mbox{se } 0<x<L \mbox{ Zona II} \\
 - \frac{\hbar^2}{2m} \partial_x^2, & \mbox{se } x>L \mbox{ Zona III}
\end{cases}$$
Descriviamo ancora una volta la funzione d'onda della nostra particella:
\begin{equation}
\begin{aligned}
  \quad \mbox{Zona I: }   &\psi = Ae^{ikx} + Be^{-ikx} \qquad     & con \qquad & k\hbar = \sqrt{2mE}\\
  \quad \mbox{Zona II: }  &\psi = A'e^{ik'x} + B'e^{-ik'x} \qquad & con \qquad & k'\hbar = \sqrt{2m(E-V)}\\
  \quad \mbox{Zona III: } &\psi = A''e^{ikx} + B''e^{-ikx} \qquad & con \qquad & k\hbar = \sqrt{2mE}
\end{aligned}
\end{equation}
Indichiamo con \myindex{onda entrante} un onda che si avvicina ad un bersaglio mentre \myindex{onda uscente} un onda che si allontana dal bersaglio. Classicamente nella \textit{Zona I} ci aspettiamo che l'onda entrante sia uguale all'onda uscente della \textit{Zona I} se $V > E$ oppure uguale a quella uscente dalla \textit{Zona III}. Nel nostro caso questo non è scontato, anzi bisogna analizzare una sovrapposizione di tutte le varie combinazioni. Gli unici vincoli che possiamo sfruttare sono le continuità nelle zone di contatto $ x = 0, x = L$ che sono: la continuità dell'ampiezza
$$A + B = A' + B'$$
$$A'e^{-\kappa L} + B'e^{\kappa L} = A''e^{ikL} + B''e^{-ikL}$$
e la continuità di pendenza
$$ikA - ikB = -\kappa A' + \kappa B'$$
$$-\kappa A'e^{-\kappa L} + \kappa B'e^{\kappa L} = ikA''e^{ikL} - ikB''e^{-ikL}$$
Da questo punto possiamo calcolarci le probabilità di rifflessione e di trasmissione dell'onda.
La \myindex{probabilità di riflessione} è data dal rapporto tra fra la densità di flusso riflessa e la densità di flusso incidente:
$$R = \frac{|B|^2}{|A|^2}$$
Analogamente la \myindex{probabilità di trasmissione} è data dal rapporto fra densità di fusso trasmessa e densità di flusso incidente:
$$P = \frac{|A''|^2}{|A|^2}$$
Si fa notare che $P = 1 - R$. Nonostante $E < V$ la particella può avere una $P \ne 0$ ovvero avere una certa probabilità di attraversare la barriera avendo comunque un energia inferiore ad essa, questa caratteristica prende il nome di \myindex{effetto tunnel}.

\subsubsection{Particella nella scatola}
\myindex{pippo}
% \subsubsection{La soluzione dell'oscillatore}
