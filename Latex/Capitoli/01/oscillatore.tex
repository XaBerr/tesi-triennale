\subsection{Oscillatore armonico}
\subsubsection{Moto traslazionale}
Il moto più semplice da introdurre è quello di una particella che si muove in uno spazio monodimensionale a potenziale costante. L'hamiltoniana di questa particella è:
$$\hat{H} = - \frac{\hbar^2}{2m} \partial_x^2$$
L'equazione di Scr\"odinger associata indipendente del tempo $\hat{H}\psi = \hat{E}\psi$ ha come soluzione:
$$\psi = A e^{i k x} + B e^{- i k x} \qquad k = \sqrt{\frac{2 m E}{\hbar^2}}$$
I coefficenti $A, B$ sono indicativi della direzione di moto della particella, ad esempio se la particella viaggia da destra a sinistra avrà sicuramente il coefficente $A = 0$, invece se viaggia da sinistra a destra avrà il coefficente $B = 0$.
Dal coefficente $k$ possiamo invece notare che:
$$k\hbar = \frac{\hbar}{\hbar} \sqrt{2 m E \hbar} = \frac{\hbar}{\hbar} \sqrt{2 m \frac{p^2}{2 m}} = p \qquad con \qquad E = \frac{p^2}{2m}$$
Riscrivendo la funzine di $\psi$ come
$$\psi = C \cos{kx} + D \sin{kx}$$
ovvero come
$$\psi = C \cos{\frac{p x 2 \pi}{h}} + D \sin{\frac{p x 2 \pi}{h}}$$
ne consegue che la lunghezza d'onda è $\lambda = \frac{2\pi}{k}$ e
$$p = \frac{h}{\lambda}$$
Quest'ultima relazione prende il nome di \index{relazione di de Broglie}\textbf{relazione di de Broglie}.

\subsubsection{Densità di flusso}
Data una funzione d'onda $\Psi$ si definisce densità di flusso:
$$J_x = \frac{\hbar}{2mi}\left(\Psi^{\star} \frac{\partial \Psi}{\partial x} - \Psi \frac{\partial \Psi^{\star}}{\partial x}  \right)$$

\subsubsection{Pacchetto d'onde}
Un pacchetto d'onde è uno stato di energia imprecisa descrivibile dalla sovrapposizione di energie precise.
La funzione d'onda risulterà di ampiezza non nulla in una piccola regione dello spazio e nulla in qualunque altro posto.
Avrà come forma
$$\Psi(x, t) = \int g(k) \Psi_k(x, t) \, dk \qquad con \qquad \Psi_k(x, t) = A e^{ikx} e^{\frac{-i E_k t}{\hbar}}$$
dove $g(k)$ prenderà il nome di \index{funzione peso}\textbf{funzione peso} e $\Psi_k$ sarà la famiglia d'onde che si sovrappongono per formare $\Psi$.

\subsubsection{Barriera di potenziale}
Classicamente una particella può superare una barriera di potenziale solo se la sua energia iniziale è superiore a quella del potenziale, in caso contrario la particella verrebbe riflessa. In meccanica quantistica la faccenda è più complessa. Vediamo ora come si comporta la particella a seconda che lo spessore della barriera sia infinito o finito.

\paragraph{Parete di potenziale infinitamente spessa}
Prendiamo una regione dello spazio e dividiamola in due, la parte sinistra $(x < 0)$ che chiameremo \textit{Zona I} decidiamo avere potenziale $0$, mentre la parte destra $(x > 0)$ la nomineremo \textit{Zona II} ed avrà potenziale $V$. L'hamiltoniana dello spazio sarà:
$$H=\begin{cases} - \frac{\hbar^2}{2m} \partial_x^2, & \mbox{se } x<0 \mbox{ Zona I} \\ - \frac{\hbar^2}{2m} \partial_x^2 + V, & \mbox{se } x>0 \mbox{ Zona II}
\end{cases}$$
Consideriamo ora una particella libera che si muove da sinistra a desta con energia iniziale $E$, descriviamo la sua funzione d'onda nelle due zone:
\begin{equation}
\begin{aligned}
  \quad \mbox{Zona I: }  &\psi = Ae^{ikx} + Be^{-ikx} \qquad & con \qquad & k\hbar = \sqrt{2mE}\\
  \quad \mbox{Zona II: } &\psi = Ae^{ik'x} + Be^{-ik'x} \qquad & con \qquad & k'\hbar = \sqrt{2m(E-V)}
\end{aligned}
\end{equation}

\paragraph{Parete di potenziale di spessore finito}
% \paragraph{Parete di potenziale di Eckart}


% \subsubsection{Particella nella scatola}
% \subsubsection{La soluzione dell'oscillatore}
