\subsection{Operatori}
Si dice \textbf{osservabile} qualunque variabile dinamica che si presti ad essere osservata. Nella meccanica classica le osservabili si rappresentano come funzioni, ad esempio la posizione in funzione del tempo $x(t) = x_0 + v_0 * x + \frac{1}{2} a t^2$, mentre nella meccanica quantistica le osservabili vengono rappresentate mediante \textbf{operatori}, la posizione applicata alla funzione d'onda $\hat{X}\psi$. Noi rappresenteremo l'operatore generico con il simbolo $\hat{\Omega}$, mentre adopereremo $\hat{A}, \hat{B}, ...$ per riferirci ad una serie di operatori. In generale distingueremo fra l'osservabile e l'operatore ad esso asociato, useremo dunque per riferirci alla posizione di una particella lungo l'asse $x$ il simbolo $x$, mentre per riferirci al suo operatore useremo $\hat{X}$.

\subsection{Operatori comuni}
Scegliendo in maniera diversa gli operatori corrispondenti ad una osservabile si ottengono rappresentazioni differenti. Una delle rappresentazioni più comuni è la \textbf{rappresentazione delle posizioni} che si distingue, appunto, per il fatto che rappresenta l'operatore posizione mediante la moltiplicazione per $x$. Esponiamo ora tre operatori importanti per la meccanica quantistica nel loro caso monodimensionale. L'\textbf{operatore posizione} sara dunque:
$$\hat{x} f = x f$$
Un altro operatore importante è l'\textbf{operatore momento}:
$$\hat{p} f = \frac{\hbar}{i} \partial_x f$$
L'operatore energia totale del sistema si chiama \textbf{operatore hamiltoniano}. Prendiamo l'energia di una particella libera $H = T + V$ dove $T$ corrisponde all'energia cinetica posseduta dalla particella $T = \frac{p^s}{2 m}$, mentre $V$ corrisponde all'energia potenziale in funzione della posizione $V(x)$. Sostituiamo i valori con i corrispondenti operatori e definiamo così l'operatore hamiltoniano:
$$\hat{H} f = \frac{1}{2}\frac{\hat{p}^2}{m} f + V(\hat{x}) f = -\frac{\hbar^2 \partial_x^2 f}{2 m} + V(\hat{x}) f$$

\subsubsection{Autofunzioni e autovalori}
In termini piani, un operatore, è un ente matematico che opera su di una funzione della $x$, trasformandola in un'altra funzione. In certi casi il risultato dell'operatore è la stessa funzione di partenza moltiplicata per un valore. Queste funzioni prendono il nome di \textbf{autofunzioni} dell'operatore.
$$\hat{\Omega} f_a = a f_a$$
In questo esempio $f_a$ è l'autofunzione di $\hat{\Omega}$ mentre $a$ è il suo \textbf{autovalore}. La funzione $e^{ax}$ è un'autofunzione dell'operatore $\hat{\partial{}_x}$ con autovalore $a$, mentre la funzione $e^{ax^2}$ non è un autofunzine dello stesso operatore. In particolare una funzione generica $g$ si può sviluppare in funzione di autofunzioni $f_n$ dell'operatore $\hat{\Omega}$, questo insieme è chiamato \textbf{insieme completo} di funzioni. Per esempio:
$$\hat{\Omega} f_n = \omega_n f_n$$
$$g = \sum_n c_n f_n$$
$$\hat{\Omega} g = \sum_n c_n \hat{\Omega} f_n = \sum_n c_n \omega_n f_n$$
Un caso particolare di queste combinazioni lineari è quando abbiamo a che fare con autofunzioni che condividono lo stesso autovalore, in questo caso si chiamano autofunzioni \textbf{degeneri}.
$$\sum_n c_n \hat{\Omega} f_n = \sum_n c_n \omega f_n = \omega \sum_n c_n f_n$$

\subsubsection{Commutatore}
Una cosa interessante degli operatori è che l'ordine in cui una successione di operatori agiuscono su una funzione influisce sul risultato. Dati due opertori $\hat{A}, \hat{B}$ indichiamo la successione $\hat{A}$ seguita da $\hat{B}$ come $\hat{B}\hat{A}$ mentre, $\hat{B}$ seguita da $\hat{A}$ come $\hat{A}\hat{B}$.
In generale si dice che gli operatori non commutano, ovvero $\hat{A}\hat{B} \ne \hat{B}\hat{A}$. La quantità $\hat{A}\hat{B} - \hat{B}\hat{A}$ prende il nome di \textbf{commutatore} e si indica con il simbolo $[\hat{A},\hat{B}]$.

\subsubsection{Lineari}
Gli operatori che troviamo nella meccanica quantistica hanno tutti una natura lineare. Per \textbf{operatore lineare} si intende quell'operatore che rispetta i canoni di linearità. Dati $a, b$ due costanti, $f, g$ due funzioni e $\hat{\Omega}$ operatore lineare, sarà vero che:
$$\hat{\Omega}( af + bg ) = a \hat{\Omega} f + b \hat{\Omega} g$$

\subsubsection{Hermitiani}
Dalla misurazione di un'osservazione non può che risultare un numero reale. Gli operatori che hanno dei numeri reali come autovalori si dicono \textbf{operatori hermitiani}. Sono dunque hermitiani gli operatori usati nel misurare un'osservabile. Matematicamente diciamo che: date due funzioni qualsiasi $\psi_m, \psi_n$, $\hat{\Omega}$ è un operatore hermitiano se soddisfa la seguente relazione
$$\int \psi_m^{\star} \hat{\Omega} \psi_n\, d\tau = \left \{ \int \psi_n^{\star} \hat{\Omega} \psi_m\, d\tau \right \}^{\star}$$
