\subsection{Operatori}
Si dice \textbf{osservabile} qualunque variabile dinamica che si presti ad essere osservata. Nella meccanica classica le osservabili si rappresentano come funzioni, ad esempio la posizione in funzione del tempo $x(t) = x_0 + v_0 * x + \frac{1}{2} a t^2$, mentre nella meccanica quantistica le osservabili vengono rappresentate mediante \textbf{operatori}, la posizione applicata alla funzione d'onda $\hat{X}\psi$. Noi rappresenteremo l'operatore generico con il simbolo $\hat{\Omega}$, mentre adopereremo $\hat{A}, \hat{B}, ...$ per riferirci ad una serie di operatori. In generale distingueremo fra l'osservabile e l'operatore ad esso asociato, useremo dunque per riferirci alla posizione di una particella lungo l'asse $x$ il simbolo $x$, mentre per riferirci al suo operatore useremo $\hat{X}$.
\subsubsection{Autofunzioni e autovalori}

\subsubsection{Commutatore}
% \paragraph{Processore}
% \subparagraph{Architettura}
ciao
$$ \epsilon = V * I $$
% \includegraphics[scale=0.5]{Immagini/icon.png}
\subsubsection{Lineari}
% \paragraph{Processore}
% \subparagraph{Architettura}
ciao
$$ \epsilon = V * I $$
% \includegraphics[scale=0.5]{Immagini/icon.png}
\subsubsection{Hermitiani}
% \paragraph{Processore}
% \subparagraph{Architettura}
ciao
$$ \epsilon = V * I $$
% \includegraphics[scale=0.5]{Immagini/icon.png}
