\lvli{Conclusioni}
\lvlii{Quantum speedup}
L'interesse nei quantum computer è legato alla loro capacità di essere molto più veloci dei computer tradizionali a risolvere un determinato tipo di problemi, questa capacità prende il nome di \idx{quantum speedup}. Al livello matematico è possibile definirlo\cite{DDQS} come il limite del rapporto
$$S = \lim_{N \to \infty} \frac{C(N)}{Q(N)}$$
tra il tempo $C(N)$ neccessario per un computer classico per risolvere un problema di dimensioni $N$ e $Q(N)$ il tempo per un computer quantistico per risolvere lo stesso problema. Questa definizione non richiede la capacità di un quantum computer di ridurre un problema da esponenziale a polinomiale come anticipato nell'introduziome ma valuta solo la capacità si scaling con l'aumentare delle dimensioni del problema.

Per analizzare il quantum speedup del D-Wave è stato scelto di confrontarlo con due algoritmi classici molto simili SA e SQA, nella ricerca del ground state attraverso un modello di Ising a due dimensioni dello spin glass.
La scelta di tale problema è dipesa dalla capacità di essere rimappato in un qualsiasi altro problema np-hard. Per il calcolo di $Q(N)$ sono state fatte delle simulazioni modificando vari parametri nel D-Wave One e nel D-Wave Two e i loro tempi sono stati valutati come $T_{DW}(N) = Q(N)$. Per quanto riguarda i computer classici sono state eseguite delle simulazioni per ogni $C(N)$ le cui prestazioni sono state misurate compensando la componente parallela\cite{EQA} come
 $\frac{T_{SA-SQA}(N)}{N} \simeq C(N)$. Da qui è stata definita\cite{DDQS} la realzione:
$$S_{DW} = \lim_{N \to \infty} \frac{T_{SA-SQA}(N)}{N \cdot T_{DW}(N)}$$

Sono state fatte delle comparazioni tra SA e DW\cite{DDQS} e tra SQA e DW\cite{QVC}. Ognuna delle comparazioni ha dimostrato che il DW dopo un certo numero $N_{crit}$ di spin inizia ad avere prestazioni inferiori rispetto ad entrambi gli algoritmi, questa è un evidente prova della \idx{mancanza di speedup} ad oggi dei sistemi testati. La mancanza di speedup non vuol dire che l'architettua così pensata non possa presentare speedup in futuro, infatti gli ottimi risultati ottenuti dal DW sotto $N_{crit}$ fanno pensare che il quantum speedup non sia raggiunto a causa della difficile calibrazione e rimozione dei rumori termici per grandi $N$ come spiegato nel capitolo dell'architettura.

Nonostante il fallimento nel dimostrare lo speedup si è scoperto un'altro legame moltro interessante, quello tra il SQA e il DW. Tale algoritmo approssima bene il comportamento per $N$ molto piccoli e per $N$ molto grandi\cite{EQA}, in particolare la sua versione continua\cite{QVC}. È possibile dividere il SQA in due varianti: quella \idx{DT-SQA} discreta definita nel capitolo sugli algoritmi e quella \idx{CT-SQA} continua che è ottenuta facendo più simulazioni DT-SQA per valori differenti di $\Delta_\tau = \frac{\beta}{M}$ ed estrapolando il valore per $\Delta_\tau \to 0$\cite{EQA,QVC} dove $\Delta_\tau$ è l'intervallo di discretizzazione del tempo $\tau$ di annealing ottenuto dividendo l'inverso della temperatura $\beta$ per il numero di livelli Suzuki-Trotter $M$.

Il CT-SQA si è rivelato avere prestazioni molto inferiori rispetto al DT-SQA visto che il calcolo usato per ottenerlo attraverso la media delle energie introduce un'energia residua. Questa osservazione potrebbe essere anche il sintomo che il QA non riesca mai a raggiungere le prestazioni del DT-SQA per colpa proprio di energie residue dovute alla continuita e non dall'effetto termico.

\begin{figure}[htbp]
\centering
\includegraphics[scale=0.6]{Immagini/residua.jpg}
\caption{Energie residue nell'esecuzione degli algoritmi.}
\label{figura:residua}
\end{figure}
Tutti gli esperimenti realizzati fino ad ora sono stati eseguiti su un modello di Ising a due dimensioni con condizioni al contorno periodiche; si augura che con il miglioramento dell'architettura del DW sia possibile confrontare modelli a tre dimensioni o superiori e che possano avere risultati differenti.
