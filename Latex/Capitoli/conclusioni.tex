\lvli{Conclusioni}
\lvlii{Quantum speedup}
L'interesse nei quantum computer è legato alla loro capacità di essere molto più veloci dei computer tradizionali a risolvere un determinato tipo di problemi, questa capacità prende il nome di \idx{quantum speedup}. Al livello matematico è possibile definirlo\cite{DDQS} come il limite del rapporto
$$S = \lim_{N \to \infty} \frac{C(N)}{Q(N)}$$
tra il tempo $C(N)$ neccessario per un computer classico per risolvere un problema di dimensioni $N$ e $Q(N)$ il tempo per un computer quantistico per risolvere lo stesso problema. Questa definizione non richiede la capacità di un quantum computer di ridurre un problema da esponenziale a polinomiale come anticipato nell'introduziome ma valuta solo la capacità si scaling con l'aumentare delle dimensioni del problema.

Per analizzare il quantum speedup del D-Wave è stato scelto di confrontarlo con due algoritmi classici molto simili SA e SQA, nella ricerca del ground state attraverso un modello di Ising dello spin glass.
La scelta di tale problema è dipesa dalla capacità di essi di essere rimappati un qualsiasi problema np-hard. Per il calcolo di $Q(N)$ sono state fatte delle simulazioni modificando varie parametri nel D-Wave One e nel D-Wave Two e i loro tempi sono stati valutati come $T_{DW}(N) = Q(N)$. Per quanto riguarda i computer classici è sono state eseguite delle simulazioni parallele per ogni $C(N)$ le cui prestazioni sono state misurate compensando la componente parallela\cite{EQA} come
 $\frac{T_{SA-SQA}(N)}{N} \simeq C(N)$. Da qui è stata definita\cite{DDQS} la realzione:
$$S_{DW} = \lim_{N \to \infty} \frac{T_{SA-SQA}(N)}{N \cdot T_{DW}(N)}$$

Sono state fatte delle comparazioni tra SA e DW\cite{DDQS} e tra SQA e DW\cite{QVC}. Ognuna delle comparazioni ha dimostrato che il DW dopo un certo numero $N_{crit}$ di spin inizia ad avere prestazioni inferiori rispetto ad entrambi gli algoritmi, questa è un evidente prova della \idx{mancanza di speedup} ad oggi dei sistemi testati. La mancanza di speedup non vuol dire che l'architettua così pensata non possa presentare speedup in futuro, infatti gli ottimi risultati ottenuti dal DW sotto $N_{crit}$ fanno pensare che il quantum speedup non sia raggiunto a causa della difficile calibrazione e rimozione dei rumori termici per grandi $N$ come spiegato nel capitolo dell'architettura.

Nonostante il fallimento nel dimostrare lo speedup si è scoperto un'altro legame moltro interessante, quello tra il CT-SQA e il DW. È possibile dividere il SQA in due varianti: quella DT-SQA discreta definita nel capitolo sugli algoritmi e quella CT-SQA continua che è ottenuta



% Defining and detecting quantum speed up
% Definizione di quantum speedup S(N).
% Si cerca lo speedup comparando algoritmi simili.
% Per i problemi semplici c'è per quelli complessi no (108q).
% SA meglio di SQA.
% Magari:
% C'è troppo errore termico
% Ci sono errori di calibrazione
% Non ha speedup

%Evidence for quantum annealing
% DW si comporta molto di più come SQA rispetto SA e SD.
% sqrt N viene mappato in N per migliorare le prestazioni.
% DW ha un t_a alto.
% La divisione per N serve per eliminare il vantaggio del parallelismo.

% Quantum vs classical annealing
% SQA è superiore al SA.
% QA non ha speedup.
% DT-SQA è superiore al CT-SQA.
% CT-SQA ha un energia residua superiore.
% CT-SQA serve per simulare il QA.
% è possibile che il QA sia più veloce del CT-SQA.
