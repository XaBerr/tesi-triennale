\lvli{Comparazione per algoritmi di ottimizzazione}
\lvlii{Simulated annealing}
\lvliii{Modello di ising}
Il modello di Ising (IM) a due dimensioni è possibile immaginarlo come una griglia $N \times N$ composta da \textit{nodi}, che rappresentano gli atomi, e \textit{archi}, che rappresentano le interazioni tra essi. Ogni nodo può assumere come valori $1$ in caso l'atomo sia spin-up o $-1$ in caso sia spin-down. Esistono due varianti del modello IM quella di Sherrington-Kirkpatrick (SK) dove l'interazione di ogni particella ha raggio infinito e quello di  Edwards-Anderson (EA) dove l'interazione ha raggio finito. In questa discussione faremo sempre riferimento al modello EA con raggio unitario, ovvero le particelle interaggiranno soltanto con quelle adiacenti. A questo modello sarà associata una funzione Hamiltoniana:
$$ H(\vec{a}, \vec{b}, \vec{q}) = - \sum_{i \in A } a_i q_i - \frac{1}{2} \sum_{i \in A} \sum_{j \in B_i} b_{ij} q_i q_j $$
in cui $A$ è l'insieme dei nodi e $B_i$ è l'insieme dei vicini del nodo $i$. In particolare $a_i$ è il valore di energia associata al nodo $i$, $q_i$ è il valore dello spin del nodo $i$ e $b_{ij}$ è il valore di energia associato alla coppia di vicinanza dei nodi $i$ e $j$. È importante tenere conto che l'hamiltoniana descritta non fa riferimento al numero di dimensioni, infatti questo modello può essere astratto a $n$ dimensioni.
\lvliii{Algoritmo}
Il primo step del processo del simulated annealing consiste nel determinare uno stato di
partenza. Solitamente questo avviene in maniera casuale e, perciò, la condizione iniziale può anche essere \textit{non ammissibile}. L’idea alla base dell’algoritmo è quella di arrivare a valutare un sottoinsieme delle soluzioni ammissibili partendo dallo stato iniziale scelto e applicando una successione di piccole perturbazioni.

L'algoritmo prevede un loop di $n$ iterazioni che simula l'abbassamento della temperatura da $t_{max}$ (temperatura di partenza) a $t_{min}$ (temperatura di arrivo) con uno scalino di $t_{delta}$. Dati quindi i parametri iniziali $t_{max}, t_{min} \in \mathbb{R}_{\geq 0}$ e $ t_{delta} \in \mathbb{R}_{> 0}$, possiamo ricavarci il valore di $n$:
$$n = \frac{t_{max}-t_{min}}{t_{delta}}$$
con $n \in \mathbb{N}$ e $t_{max} > t_{min}$.

Ad ogni iterazione del ciclo viene generata una soluzione \textit{vicina} a quella \textit{attuale} e viene deciso quale delle due tenere.

Un esempio di criterio di vicinanza può essere espresso attraverso il peso di Hamming: supponiamo di avere una sequenza di 8 bit $00001111$ con peso di Hamming 4, se noi scegliamo 2 come peso di Hamming massimo di vicinanza, allora una configurazione vicina può essere $00111111$.

Lo step successivo è l'applicazione dal criterio di Metropolis, che rappresenta il cuore dell'algoritmo: in questo passo viene scelto se tenere la configurazione \textit{attuale} o la configurazione \textit{vicina}. Se la configurazione \textit{vicina} è migliore di quella \textit{attuale} viene certamente scelta quella \textit{vicina}, contrariamente se quella \textit{attuale} è meglio di quella \textit{vicina} si sceglierà di tenere quella più svantaggiosa delle due (\textit{vicina}) in maniera semi-casuale.
La semi-casualità viene data dalla generazione di un numero casuale $k \in [0,1]$ e dal confrontato con la distribizione di Boltzman di parametri: $t = t_{max} - t_{delta} * n$ e $\Delta E = |E_{vicina}| - |E_{attuale}|$.
$$B(\Delta E, t) = e^{- \frac{\Delta E}{k_B * t}}$$
Questo significa che, in corrispondenza di temperature alte il sistema avrà la possibilità di muoversi liberamente di configurazione in configurazione, mentre all’abbassarsi della temperatura la sua probabilità di accettare soluzioni peggiori diminuirà.

In generale si può vedere l'algoritmo SA come una procedura che mira a minimizzare la configurazione di energia di un sistema dato. Il modello associerà una variabile $\alpha_i \in \mathbb{R}$ modificabile ad ogni costante $\beta_i \in \mathbb{R}$, che rappresenta l'energia per ogni punto del sistema. Intuitivamente si può pensare allo scopo della procedura come quello di trovare gli $\alpha_i$ in modo tale da minimizzare il valore $\alpha_i * \beta_i$. Chiaramente questo è solo un'esempio, la funzione di enegia dipende dal modello di riferimento e quindi può cambiare.

\lvlii{Simulated quantum annealing}
\lvliii{Introduzione}
L'introduzione del simulated quantum annealing (SQA) avviene per migliorare l'ergodicità del sistema, ovvero l'indipendenza della soluzione del problema dallo stato iniziale. Esistono due tipi di ergodicità quella \textit{debole} che delinea una perdita di memoria dello stato iniziale man mano che si evolve il sistema e quella \textit{forte} che prevede la convergenza ad una soluzione indipendentemente dallo stato di partenza. Il SQA migliora l'ergodicità nel caso in cui ci siano due minimi locali divisi da una \textit{spike} (punta di potenziale molto alta e di base stretta) perché la probabilità di transizione da un minimo all'altro è superiore se si sfrutta il \textit{tunneling}(SQA) piuttosto che per \textit{salto} termico(SA).

\lvliii{Modello di ising trasverso}
Il modello di ising trasverso (TIM) nasce per rappresentare l'effetto quantistico che è presente in un insieme di atomi, è possibile immaginarlo come un modello di ising tradizionale con l'aggiunta di una componente quantistica. Prendiamo di nuovo in esempio il modello di ising a due dimensioni dove ogni atomo questa volta oltre ad essere in spin-up o spin-down può essere anche in una super position dei due stati. Definiamo quindi un'asse di riferimento ad esempio l'asse x dove andremo alla fine del nostro esperimento a misurare il valore degli spin. Una volta che decideremo di osservare lo stato del sistema nella nostra base x il modello quantistico \textit{collasserà} in quello tradizionale e si comporterà come tale. Oltre alla base di osservazione viene introdotta un'altra base ortogonale alla prima, ad esempio l'asse z, questo asse sarà quello in cui verrà preparato il sistema nella condizione iniziale. L'idea alla base e di far in modo che il modello di ising passi dalla base z alla base x attraverso uno scheduling da noi scelto. Il coefficente che rappresenta questo scheduling è $\Gamma_{orto}$. L'emiltoniana del sistema viene così descritta:
$$ H(\vec{a}, \vec{b}, \vec{\sigma}, \vec{\Lambda}, \Gamma_{orto}) = - \sum_{i \in A } a_i \sigma^x_i - \frac{1}{2} \sum_{i \in A} \sum_{j \in B_i} b_{ij} \sigma^x_i \sigma^x_j - \Gamma_{orto} \sum_{i \in A } \Lambda_i \sigma^z_i$$
In questa versione al posto dei valori di spin troviamo le matrici di Pauli $\sigma^x_i$ e $\sigma^z_i$ per rappresentare le super position e in aggiunta dei coefficenti $\Lambda_i$ che parametrizzano l'effetto tunneling.

\lvliii{Approssimazione Suzuki-Trotter}
Il primo passo per implementare il SQA è approssimare il modello matematico quantistico attraverso un modello classico computabile da un calcolatore tradizionale. Sfruttando il formalismo Suzuki-Trotter possiamo mappare il nostro modello TIM n-dimensionale in un modello IM (n+1)-dimensionale dove la dimensione aggiuntiva rappresenta l'evoluzione temporale del sistema delle sue altre dimensioni. La procedura per ricavarci il modello IM parte dalla considerazione di dividere l'hemiltoniana $H = H_{classic} + H_{kinetic}$ nella sua parte classica di energia potenziale $H_{classic}$ e nella sua parte transversa rappresentante l'energia cinetica $H_{kinetic}$. Ora si può scrivere la funzione di partizione
$$Z = Tr (exp(-\frac{H_{classic} + H_{kinetic}}{T}))$$ e attraverso la formula di Trotter riscriverla come:
$$Z= \lim_{M \to +\infty} \sum_i \langle\sigma_i|[exp(-\frac{H_{classic}}{M \cdot T}) \cdot exp(-\frac{H_{kinetic}}{M \cdot T})]^M| \sigma_i\rangle$$
Applicandola alla funzione di partizione possiamo dividere in la parte potenziale:
$$Z_{classic} = \prod^M_{k = 1}\langle\sigma_{1,k}\cdot\cdot\cdot\sigma_{N,k}|exp(\frac{1}{M \cdot T} \sum_{i,j} J_{ij}\sigma^x_i\sigma^x_j)|\sigma_{1,k+1}\cdot\cdot\cdot\sigma_{N,k+1}\rangle = exp(\sum_{i,j=1}^N\sum_{k=1}^M \frac{J_{ij}}{M \cdot T}\sigma_{ik}\sigma_{jk})$$
dove i valori $\sigma_i = \pm 1$ sono gli autovalori dell'operatore $\sigma^x$, e la parte cinetica:
$$Z_{kinetic} = \prod^M_{k = 1}\langle\sigma_{1,k}\cdot\cdot\cdot\sigma_{N,k}|exp(\frac{\Gamma}{M \cdot T} \sum_{i}\sigma^z_i)|\sigma_{1,k+1}\cdot\cdot\cdot\sigma_{N,k+1}\rangle =$$
$$= [\frac{1}{2} sinh ( \frac{2\Gamma}{M \cdot T} )]^{\frac{N \cdot M}{2}} exp( \frac{1}{2} ln(coth(\frac{\Gamma}{M \cdot T}))\cdot\sum_{i=1}^N\sum_{k=1}^M \frac{J_{ij}}{M \cdot T}\sigma_{i,k}\sigma_{i,k+1})$$
dove in questo caso $\sigma_i = \pm 1$ sono gli autovalori dell'operatore $\sigma^z$.
A questo punto ci ricaviamo:
$$ Z \simeq Z_M = C^{NM} \sum_{z^1}\cdot\cdot\cdot\sum_{z^M} exp(\frac{-H_{d+1}}{M\cdot T})$$
ed in fine otteniamo:
$$H_{d+1} = - \sum^M_{k = 1}(\sum_{i,j} J_{i,j} \sigma^k_i \sigma^k_j + J_{orto} \sum_i \sigma^k_i \sigma^{k+1}_i)$$
con  $\sigma_i = \pm 1$ i valori di spin classici, $M$ è la dimensione aggiuntiva, $T$ la temperatura fissa del sistema, $J_{i,j} \in \mathbb{R}$ e
$$J_{orto} = - \frac{M\cdot T}{2} ln(tanh(\frac{\Gamma}{M \cdot T}))$$
$H_{d+1}$ rappresenta l'energia associata alla funzione di partizione $Z_M$ che è assintotica a $Z$ per un sistema alla temperatura $M \cdot T$.

\lvliii{Algoritmo}
SQA con PIMC.
