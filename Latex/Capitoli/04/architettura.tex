\subsection{Architettura del processore}
\subsubsection{Introduzione}
Il quantum computer della \textit{D-Wave} è un computer che si basa sul quantum anneling. Il processore ha dei qubit sviluppati con l'architettura rf-SQUID verrà specificato meglio nella sezione dedicata ai qubit ora basti sapere che sono degli anelli superconduttivi che possono essere accoppiati a due a due. Un quantum computer per eseguire algoritmi ha necessita di un numero considerevole di qubit, il \textit{D-Wave One} ne aveva 512, è un numerò così grande di qubit genera delle problematiche di costruzione sopratutto per quanto riguarda la scalabilità. Fortunatamente gran parte dell'elettronica moderna è riutilizzabile per sviluppare il primo modello il \textit{D-Wave One} basato sull'architettura dei \myindex{SFQ} super flux quanta. Da allora la tecnologia si è evoluta e con nuove tecniche e nuove architetture si è realizzato il \textit{D-Wave Two} computer di seconda generazione. L'innovazioni principali non sono sui qubit o sulla loro gestione di accoppiamento, ma sono è cambiato il sistema di controllo del processore e il sistema di fabbricazione. Il \textit{D-Wave One} usava un demultiplexer per controllare i SFQ questo implicava che per controllare $N$ qubits bisognava avere $O(\log{N})$ linee di indirizzamento. Questa architettura era stata studiata per dissipare meno energia possibile durante la programmazione. \textit{D-Wave Two} ha invece sradicato completamente il problema eliminando completamente la dissipazione di energia statica attraverso l'introduzione dell'\myindex{indirizzamento XYZ} con l'utilizzo di solo $O(\sqrt[3]{N})$ linee. Un altro obbiettivo raggiunto del \textit{D-Wave Two} è stato migliorare le prestazioni dell'algoritmo di anneling incrementando la scala energetica dei qubit. Questo obbiettivo è stato raggiunto riducendo la dimensione fisica dei qubit di un fattore di due che ha obbligato ad aumentare da quattro a sei il numero di livelli di metallo complicando il processo di fabbricazione ma ha portato un vantaggio di aumentare di quattro volte la densità del processore.

\subsubsection{Topologia}
Il processore sul quantum anneling è basato sul modello di Ising e mira a risolvere problemi di minimizzazione sui grafi, in particolare è stato pensato per risolvere questo problema: dato un grafo fisico $G$, minimizzare la seguente forma quadratica di variabili discrete $s_i \in \{-1, +1\}$:
$$E(\vec{s}|\vec{h}, \hat{J}) = \sum_{i \in nodes(G)} h_i s_i + \sum_{<i,j> \in nodes(G)} J_{ij} s_i s_j$$
con i parametri del problema che sono $h_i, J_{ij} \in {-1, -7/8, ..., +7/8, +1}$. La topologia corrispondente all'architettura hardware utilizzata nel \textit{D-Wave One} e nel \textit{D-Wave Two} è stata chiamata \myindex{Chimera}. La struttura Chimera è sata basata su tre vincoli importati: non planarità, possibilità di sviluppare grafi completi e la possibilità di avere dei circuiti di controllo. La \myindex{non planarità} è necessaria quando si vuole poter risolvere problemi np-completi, la soluzione è stata ottenuta introducendo la catena di qubit. L'\myindex{abilità di sviluppare grafi completi} non è solo legata alla non planarità ma è anche legata alla possibilità di mappare il grafo di qubit logici in qubit hardware che differiscono in termini di possibilità di interconnessione. Come anticipato in precedenza un qubit logico può essere sviluppato con quella che viene chiamata una catena di qubit fisici ma può essere rappresentato da strutture più complesse come alberi o sottografi. Per capire se un grafo può essere rappresentato all'interno dell'hardware si utilizza la \textit{straightforward prescription}.
