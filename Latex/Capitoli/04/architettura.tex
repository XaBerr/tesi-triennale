\subsection{Architettura del processore}
\subsubsection{Introduzione}
Questa sezione andrà a parlare dei primi due computer commerciali della \textit{D-Wave} che sono il \myindex{D-Wave One} e il \myindex{D-Wave Two} e andrà ad esplorare la loro architettura per comprendere l'idea che la \textit{D-Wave} ha di quantum computer. Il loro computer si basa sul quantum anneling spesso abbreviato con \myindex{QA}. Il processore ha dei qubit sviluppati con l'architettura rf-SQUID che sono degli anelli superconduttivi che possono essere accoppiati a due a due. Un quantum computer per eseguire algoritmi ha necessita di un numero considerevole di qubit, il \textit{D-Wave One} ne possiede 128 mentre il \textit{D-Wave Two} ne possiede 512. Un numerò così grande di qubit genera delle problematiche di costruzione sopratutto per quanto riguarda la scalabilità. Fortunatamente gran parte dell'elettronica moderna dell'epoca fu riutilizzabile per sviluppare il primo modello il \textit{D-Wave One} basato sull'architettura dei \myindex{SFQ} single flux quanta. Da allora la tecnologia si è evoluta e con nuove tecniche e nuove architetture si è realizzato il \textit{D-Wave Two} computer di seconda generazione. L'innovazioni principali non furono sui qubit o sulla loro gestione di accoppiamento, ma furono sul sistema di controllo del processore e sul sistema di fabbricazione. Il \textit{D-Wave One} usava un demultiplexer per controllare i SFQ, questo implicava che per controllare $N$ qubit bisognava avere $O(\log{N})$ linee di indirizzamento. Questa architettura era stata studiata per dissipare meno energia possibile durante la programmazione. Il \textit{D-Wave Two} ha invece sradicato completamente il problema, eliminando totalmente la dissipazione di energia statica attraverso l'introduzione dell'\myindex{indirizzamento XYZ} con l'utilizzo di solo $O(\sqrt[3]{N})$ linee.
Un altro obiettivo raggiunto del \textit{D-Wave Two} è stato il miglioramento delle prestazioni dell'algoritmo di anneling ottimizzando la scala energetica dei qubit, traguardo raggiunto riducendo la dimensione fisica dei qubit di un fattore di due. Questa modifica ha influito negativamente sul processo di fabbricazione obbligando ad aumentare da quattro a sei il numero di livelli di metallo ma positivamente ha aumento di quattro volte la densità del processore. Ora proseguiamo analizzando la topologia del processore che è comune per entrambe le macchine.

\subsubsection{Topologia}
Il processore sul quantum anneling è basato sul modello di Ising e mira a risolvere problemi di minimizzazione sui grafi, in particolare è stato pensato per risolvere questo problema: dato un grafo fisico $G$, minimizzare la seguente forma quadratica di variabili discrete $s_i \in \{-1, +1\}$:
$$E(\vec{s}|\vec{h}, \hat{J}) = \sum_{i \in nodes(G)} h_i s_i + \sum_{<i,j> \in nodes(G)} J_{ij} s_i s_j$$
con i parametri del problema che sono $h_i, J_{ij} \in {-1, -7/8, ..., +7/8, +1}$. La topologia corrispondente all'architettura hardware utilizzata nel \textit{D-Wave One} e nel \textit{D-Wave Two} è stata chiamata \myindex{Chimera}. La struttura Chimera è sata basata su tre \myindex{vincoli topografici} importati: non planarità, possibilità di sviluppare grafi completi e la possibilità di avere dei circuiti di controllo. La \myindex{non planarità} è necessaria quando si vuole poter risolvere problemi np-completi, la soluzione è stata ottenuta introducendo la catena di qubit. L'\myindex{abilità di sviluppare grafi completi} non è solo legata alla non planarità ma è anche legata alla possibilità di mappare il grafo di qubit logici in qubit hardware che differiscono in termini di possibilità di interconnessione. Come anticipato in precedenza un qubit logico può essere sviluppato con quella che viene chiamata una catena di qubit fisici ma può essere rappresentato da strutture più complesse come alberi o sottografi. Per capire se un grafo può essere rappresentato all'interno dell'hardware si utilizza la \textit{straightforward prescription}.
La necessità di avere dei \myindex{circuiti di controllo} diventa indispensabile con l'aumentare del numbero di qubit, infatti sebbene pochi qubit qubit possano essere controllati con precisione da linee di controllo, molti qubit presentano dei problemi risolvibili solo mediante un circuito di controllo dedicato. Attualmente il \myindex{processore QA} ha delle \textit{\"manopole\"} per controllare il sistema, in particolare ce ne sono sei associate ad ogni qubit per standardizzare le variazioni dovute alla fabbricazione e una associata agli accoppiamenti dei qubit. Per realizzare queste manopole di controllo sono stati impiegati dei bias a flusso statico programmabili da sistemi di controllo che combinano la trasformazione digitale analogia insieme alla scrittura di una memoria persistente. Questi sistemi di controllo sono stati chiamati flux DAC o \myindex{$\Phi$-DAC} (flux digital analog converter). Il loro numero considerevole all'interno del circuito influenza la forma dei qubit e la topologia del circuito.
Oltre ai vincoli della topologia ci sono anche altre costrizioni in particolare andremmo ad esporne le cinque più importanti. Dal punto di vista ideale ogni qubit dovrebbe essere connesso ad ogni altro, sfortunatamente al livello fisico c'è una limitazione di \myindex{interconnessione tra qubit} perché sebbene si possano realizzare catene, alberi, e sottografi dei qubit logici, un qubit può essere collegato solo ad un numero di qubit $\le 10$ a causa di caratteristiche non ideali e problemi delle scale energetiche introdotti dallo superamento di questo limite. Un altro vincolo che deriva dal minimizzare le scale di energia e anche introduce il problema di massimizzare la forza di accoppiamento è la \myindex{minimizzazione del disaccopiamento} ovvero minimizzare la superficie dei qubit che non viene utilizzata per gli accoppiamenti con altri qubit. Idealmente tutta la superfice dei qubit dovrebbe essere intersecata dalla superfice dei sistemi di accoppiamento e tutti i sistemi di accoppiamento dovrebbero avere la superfice intersecata con i qubit. Questa struttura composta da qubit, sistemi di accoppiamento, $\Phi$-DAC è implementata su un chip ovvero utilizza un \myindex{integrazione bidimensionale}. Sebbene sviluppare questa tecnologia attraverso un integrazione tridimensionale sarebbe più vantaggiosa dal lato teorico, dal lato pratico è molto più semplice sviluppare attraverso un sistema scalabile a due dimensioni. Qui si introduce il quinto dei nostri vincoli, ovvero, la regolarità e l'introduzione della \myindex{unità piastrella}. Proprio come accade nella vita di tutti i giorni, ricoprire un pavimento con piastrelle regolari è molto più facile e questa differenza si sente sempre di più aumentando la superfice che vogliamo ricoprire.\\
La scelta della struttura Chimera è risultata vantaggiosa per risolvere a molti di questi problemi. La prima caratteristica importante che ha questa struttura è l'utilizzo dei flux qubit. Questi qubit a flusso sono degli anelli macroscopici con delle interruzioni chiamate \textit{Josephson junction}. La loro forma macroscopica permette di deformare la forma del qubit allungandol e ruotandolo a piacere a seconda delle necessità. Stessa cosa vale molto similarmente per i sistemi di accoppiamento. Una unità piastrella Chimera è composta da otto qubit, quattro orizontali e quattro verticali disposti a griglia ed accoppiati nelle loro intersezioni. L'unità e strutturata in modo che disponendo due piastrelle adiacenti sia in orizzontale che in verticale si possano allineare i qubit e quindi anche supportare accoppiamenti di qubit esterni con altre piastelle. Questa disposizione di qubit è associabile ad un grafo non planare, ovvero ad un grafo che disteso su un piano non può essere disegnato senza incrociare gli archi. Usando l'abilità di creare catene di qubit fisici, esiste un approccio diretto per incorporare grafici completi fino ai nodi $4N$ in una griglia $NxN$ delle cellule unità indicata come $C_N$. Un modo pratico per vedere questo grafo completo è prendere una piastrella e decidere di rappresentare un qubit logico come due qubit fisici, in particolare scegliamo di prenderne uno orizzontale e uno verticale. Così facendo avremo quattro qubit totali rappresentabili all'interno della cella dove ogni qubit si collega con tutti gli altri tre formando un grafo completo. Diventa intuitivo poi generalizzare questo concetto allungando la nostra catena di quibit fisici attraverso più celle. Un altra caratteristica della struttura chimera è la possibilità di avere integrati i sitemi di controllo sulla piastrella. Ogni buco della griglia $4x4$ è occupato da tre sistemi di controllo: uno per controllare il qubit verticale uno per controllare il qubit orizzontale e uno per controllare il loro sistema di accoppiamento. I sistemi di accoppiamento hanno una forma a L che si distribisce, partendo dal punto in cui i qubit si sovrappongono, per tutto il qubit fino a poco prima del prossimo incrocio dei qubit. Questa forma permette appunto la massima copertura reciproca di superfici tra sistema di accoppiamento e qubit.
