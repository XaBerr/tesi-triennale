\lvlii{Quantum annealing}
\lvliii{Introduzione}
Il quantum annealing si distingue dal annealing termico dalla dipendenza della tempera dal tempo in cui la dinamica del sistema si blocca. La \textit{D-Wave} usata il quantum annealing per trovare lo stato di configurazione di energia minima, chiamato \idx{ground state}, in una matrice di qubit superconduttivi a flusso con accoppiamenti di spin programmabili. Questo sistema di qubit può essere programmato per realizzare una vasta gamma di differenti network di spin. Molti problemi attualmente di computazione difficile per i computer classici, nei campi che vanno dall'intelligenza artificiale alla zoologia, possono essere riformulati come il problema di trovare la più bassa configurazione energetica o stato fondamentale di un sistema di spin di Ising. La sua importanza nell'ambito scentifico non è legata solo allo sviluppo di algoritmi ma rappresenta anche il ponte di comunicazione tra lo studio teorico di network di spin isolati e le sperimentazioni pratiche sui campioni di massa magnetica.
