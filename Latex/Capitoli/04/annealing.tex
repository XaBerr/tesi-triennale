\lvlii{Quantum annealing}
\lvliii{Introduzione}
Il quantum annealing si distingue dal annealing termico dalla dipendenza della tempera dal tempo in cui la dinamica del sistema si blocca. La \textit{D-Wave} usata il quantum annealing per trovare lo stato di configurazione di energia minima, chiamato \idx{ground state}, in una matrice di qubit superconduttivi a flusso con accoppiamenti di spin programmabili. Questo sistema di qubit può essere programmato per realizzare una vasta gamma di differenti network di spin. Molti problemi attualmente di computazione difficile per i computer classici, nei campi che vanno dall'intelligenza artificiale alla zoologia, possono essere riformulati come il problema di trovare la più bassa configurazione energetica o stato fondamentale di un sistema di spin di Ising. La sua importanza nell'ambito scentifico non è legata solo allo sviluppo di algoritmi ma rappresenta anche il ponte di comunicazione tra lo studio teorico di network di spin isolati e le sperimentazioni pratiche sui campioni di massa magnetica.

\lvliii{Funzionamento}
Per crare un processore che sfrutti il quantum annealing si ha bisogno di un sistema di spin quantistico programmabile in cui potremmo controllare i singoli spin e loro accoppiamenti, eseguire l'annealing quantico e poi determinare lo stato di ogni spin.
Fino a poco tempo fa la tecnologia è stata limitata alle configurazioni realizzabili nei sistemi della materia condensata come nanomagneti molecolari o solidi di massa con comportamento quantistico critico. Sfortunatamente questi sistemi non possono essere controllati o misurati al livello individuale di spin e quindi non possono essere programmabili. La risonanza magnetica nucleare è stata utilizzata come tecnica per dimostrare un algoritmo su tre quantum bit di quantum annealing, sono stati usati tre ioni intrappolati per realizzare una simulazione quantistica di un piccolo sistema di Ising.
La soluzione utilizzata dalla \textit{D-Wave}, scelta per la sua praticità, è l'utilizzo di un sistema artificiale di Ising che sfrutta i flussi superconduttivi per realizzare i quantum bit (qbits). Un \idx{qubit} è realizzato con l'unione di due anelli uno più grande che chiameremo $a_1$ e uno più piccolo $a_2$. L'anello più piccolo vanta di due \textit{Josephson junction}. Ogni anello è influenzato da un flusso esterno di bias, rispettivamente $\phi_{x1}$ e $\phi_{x2}$. Questo sitema può essere modellato come un pozzo a potenziale doppio, di meccanica quantistica, con l'energia del sistema rappresentata da $\phi_1$. L'altezza della barriera di potenziale $\delta U$ è controllato tramite $\phi_{2x}$ mentre la distanza $2h$ tra i due minimi di energia è controllata da $\phi_{1x}$.
I due stati di energia minimi del sistema, corrispondenti al verso di rotazione orario o antiorario della cricolazione di corrente nell'anello $a_1$, vengono rinominate $\left|\uparrow \right\rangle$ e $\left|\downarrow \right\rangle$, o più comodamente $|1 \rangle$ e $|0 \rangle$. Le frecce su e giù ricordano il verso del flusso $\phi_1$ calcolato mediante la regola della mano destra. Questo modello diventa aderente a quello di Ising per temperatura molto basse dove il sistema può stare solo in uno dei due stati di minima energia. I qubit durante l'annealing tenderanno ad allinearsi a seconda dei bias programmati, in particolare i bias dei sistemi di accoppiamento servono per favorire l'allineamento o l'antiallineamento tra due qubit. Il comportamento di ogni singolo spin viene descritto attraverso l'hamiltoniana del sistema del modello di Ising:
$$H_p = \sum_{i=1}^N h_i \sigma_i^z + \sum_{i,j=1}^N J_{ij} \sigma_i^z \sigma_j^z$$
dove $\sigma_i^z$ e $\sigma_j^z$ sono le matrici di Pauli, rispetto l'asse $z$, aventi come autovettori $\{|1\rangle, |0\rangle\}$, $2h_i$ è l'energia di bias applicata al singolo spin $i$ e $2J_{ij}$ è l'energia di accoppiamentro tra i due spin $i$ e $j$. I parametri che possono essere impostati indipendentemente sono $h_i$ e $J_{ij}$. La dinamica del sistema ad annealing quantistico può essere descritta dalla funzione hemiltoniana:
$$H(t) = \Gamma(t) \sum_{i=1}^N \Delta_i \sigma_i^x + \Lambda(t) H_p$$
$\Gamma$ e $\Lambda$ sono due variabili che rappresentano la dinamica del sistema. $\Gamma$ si attenua $1 \to 0$ contemporaneamente a $\Lambda$ che si amplifica $0 \to 1$ generando un passaggio dolce tra le due componenti. La prima delle due componenti è $\sum_{i=1}^N \Delta_i \sigma_i^x$ ed esprime lo stato iniziale del sistema dove gli spin sono preparati uno ground state rispetto ad $x$ e quindi in superposition rispetto ad $z$ con $\Delta_i$ parametro di tunnelling tra $|1\rangle$ e $|0\rangle$. La seconda componete $H_p$ è invece l'hamiltoniana del nostro modello di Ising del quale ci interessa conoscere le posizioni finali $\sigma_i^z$.
Se la migrazione è fatta sufficentemente lenta il sistema rimmarrà nel ground state fino alla fine terminando in uno ground state di $H_p$.

\lvliii{Stabilire il tipo di annealing}
Il sistema qui trattato ha bisogno, per funzionare in maniera corretta, di un annealing quantistico e non di una annealing termico.
Sia l'annealing quantistico che quello termico partono da un sistema dove tutti i qubit sono in superposition, poi durante l'evoluzione vanno ad esplorare tutte le configurazioni di energia fino a trovare quella di minimo. La differenza tra i due sta nel modo in cui esplorano le configurazioni, l'annealing quantistico utilizza fluttuazioni quantistiche del tunnelling mentre quello termico utilizza fluttuazioni termiche deboli. Un modo pratico per vedere la differenza tra le due dinamiche è che durante l'evoluzione la barriera si alza gradualmente tra i due pozzi, l'anneling classico si muove tra i due ground state "saltando" sopra la barriera, mentre l'anneling quantistico si muove passando attraverso la barriera grazie al tunneling.
Se la fluttuazione termica è dominante, la dinamica del qubit può essere considerata come un attivazione termica (salto) sopra la barriera con un tasso proporzionale a $e^{-\frac{\delta U}{k_B T}}$ dove $T$ è la temperatura e $k_B$ è la costante di Boltzmann. Da questa formula si deduce che la dinamica si interrompe quando $\delta U >> k_B T$. Sapendo che $\delta U$ si incrementa con il tempo, possiamo definire un tempo di blocco del sistema $t_{freeze}^{TA}$ dove $\delta U(t_{freeze}^{TA}) \approx k_bT$ e dove $TA$ sottointende \textit{Termal Annealing}. Nell'intervallo di tempo in questione, essendo $\delta U$ quasi lineare rispetto al tempo, comporta $t_{freeze}^{TA}$ ad essere linearmente dipendente dalla temperatura $T$.
Diversamente se la fluttuazione quantistica è dominante sarà l'effetto tunnel a diminuire con l'alzarsi della barriera $\delta U$. Ci sarà questa volta un altro tempo $t_{freeze}^{QA}$ indipendente dalla temperatura $T$ (debolmente dipendente) dove $QA$ sottointende \textit{Quantum Annealing}. Misurando la dipendenza di $T$ da $t_{freeze}$ possiamo determinare qual'è la componente dominante nell'annealing.
Si realizza ora un esperimento creato su misura per determinare il tipo di fluttuazione dominante. Si parte cercando di determinare il tempo $t_freeze$ nel sistema di spin. Per far ciò si vuole fare è misurare la risposta a gradino del sistema a cambiamenti rapidi di $h$, rispetto le grandezze di $\Gamma$ e $\Lambda$, a stage differenti durante il quantum annealing. Dopo un intervallo di tempo $t_d$ si varierà la funzione $h(t)$ da $0 \to h_t$ e si misurerà la probabilità $P_1$ di trovare il sitema nello stato $|1\rangle$. Le dinamiche che si ottengono sono due a seconda che $h$ sia fatta variare presto o tardi nella scala dei tempi rispetto a $t_{freeze}$. Se $h$ viene fatto variare presto, quando $\delta U$ è ancora bassa, la probabilità misurata sarà $P_1 \ge 0.5$. In questo caso l'effetto termico sarà dominante e il risultato dipenderà dalla distribuzione di Boltzmann. Invece se $h$ viene fatta variare dopo che la barriera è sufficentemente alta $t_d > t_{freeze}$ il sistema si troverà in uno stato di superposition e la probabilità varrà $P_1 \approx 0.5$.
Facendo variare la temperatura $T$ e misurando il tempo $t_{freeze}$ si rileva una saturazione di $t_{freeze}$ sotto la temperatura di $T_{threshold}$. Poiché la probabilità $P_1$ non satura a $T_{threshold}$ dove $t_{freeze}$ satura, è un chiaro indicatore che la saturazione di $t_{freeze}$ non è un dipende dalla saturazione della temperatura $T$ del qubit. La conclusione è che il sistema diventa di tipo quantomeccanico per temperature molto basse sotto una certa soglia $T_{threshold}$.
