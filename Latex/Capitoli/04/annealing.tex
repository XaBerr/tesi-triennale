\lvlii{Quantum annealing}
\lvliii{Introduzione}
Il quantum annealing si distingue dal annealing termico dalla dipendenza della tempera dal tempo in cui la dinamica del sistema si blocca. La \textit{D-Wave} usata il quantum annealing per trovare lo stato di configurazione di energia minima, chiamato \idx{ground state}, in una matrice di qubit superconduttivi a flusso con accoppiamenti di spin programmabili. Questo sistema di qubit può essere programmato per realizzare una vasta gamma di differenti network di spin. Molti problemi attualmente di computazione difficile per i computer classici, nei campi che vanno dall'intelligenza artificiale alla zoologia, possono essere riformulati come il problema di trovare la più bassa configurazione energetica o stato fondamentale di un sistema di spin di Ising. La sua importanza nell'ambito scentifico non è legata solo allo sviluppo di algoritmi ma rappresenta anche il ponte di comunicazione tra lo studio teorico di network di spin isolati e le sperimentazioni pratiche sui campioni di massa magnetica.

\lvliii{Funzionamento}
Per crare un processore che sfrutti il quantum annealing si ha bisogno di un sistema di spin quantistico programmabile in cui potremmo controllare i singoli spin e loro accoppiamenti, eseguire l'annealing quantico e poi determinare lo stato di ogni spin.
Fino a poco tempo fa la tecnologia è stata limitata alle configurazioni realizzabili nei sistemi della materia condensata come nanomagneti molecolari o solidi di massa con comportamento quantistico critico. Sfortunatamente questi sistemi non possono essere controllati o misurati al livello individuale di spin e quindi non possono essere programmabili. La risonanza magnetica nucleare è stata utilizzata come tecnica per dimostrare un algoritmo su tre quantum bit di ricottura quantistica, sono stati usati tre ioni intrappolati per realizzare una simulazione quantistica di un piccolo sistema di Ising.
La soluzione utilizzata dalla \textit{D-Wave}, scelta per la sua praticità, è l'utilizzo di un sistema artificiale di Ising che sfrutta i flussi superconduttivi per realizzare i quantum bit (qbits). Un \idx{qubit} è realizzato con l'unione di due anelli uno più grande che chiameremo $a_1$ e uno più piccolo $a_2$. L'anello più piccolo vanta di due \textit{Josephson junction}. Ogni anello è influenzato da un flusso esterno di bias, rispettivamente $\phi_{x1}$ e $\phi_{x2}$. Questo sitema può essere modellato come un pozzo a potenziale doppio, di meccanica quantistica, con l'energia del sistema rappresentata da $\phi_1$. L'altezza della barriera di potenziale $\delta U$ è controllato tramite $\phi_{2x}$ mentre la distanza $2h$ tra i due minimi di energia è controllata da da $\phi_{1x}$.
I due stati di energia minimi del sistema, corrispondenti al verso di rotazione orario o antiorario della cricolazione di corrente nell'anello $a_1$, vengono rinominate $\left|\uparrow \right\rangle$ e $\left|\downarrow \right\rangle$, o più comodamente $|1 \rangle$ e $|0 \rangle$. Le frecce su e giù ricordano il verso del flusso $\phi_1$ calcolato mediante la regola della mano destra. Questo modello diventa aderente a quello di Ising per temperatura molto basse dove il sistema può stare solo in uno dei due stati di minima energia. I qubit durante l'annealing tenderanno ad allinearsi a seconda dei bias programmati, in particolare i bias dei sistemi di accoppiamento servono per favorire l'allineamentoo o l'antiallineamento tra due qubit. Il comportamento di questo sistema viene descritto attraverso l'hamiltoniana del modello di Ising:
$$H_p = \sum_{i=1}^N h_i \sigma_i^z + \sum_{i,j=1}^N J_{ij} \sigma_i^z \sigma_j^z$$
dove $\sigma_i^z$ e $\sigma_j^z$ sono le matrici di Pauli aventi come autovettori $\{|1\rangle, |0\rangle\}$, $2h_i$ è l'energia di bias applicata al singolo spin $i$ e $2J_{ij}$ è l'energia di accoppiamentro tra i due spin $i$ e $j$.
