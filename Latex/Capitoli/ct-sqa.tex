\section{CT-SQA}
Il continuos time simulate quantum annealing è spesso chiamato Swendsen-Wang è un'algorimto che cerca di ottenere il valore minimo di energia per IM attraverso a dei raggruppamenti di spin in cluset.

Ancora una volta per applicare un algoritmo quantistico in uno tradizionale bisogna approssimare il modello TIM n-dimensionale in un modello IM (n+1)-dimensionale attraverso l'approssimazione Suzuki-Trotter con condizioni al contorno periodiche; ottenendo la solita funzione classica di energia:
$$S_{class} = - \sum_{\tau,\langle ij \rangle} K_{ij} S_i(\tau) S_j(\tau) - \sum_{\tau, i} K_{ij}' S_i(\tau) S_i(\tau+1)$$
che questa volta viene utilizzata per esprimere l'energia libera del sistema:
$$F = - T \cdot \lim_{\Delta \tau \to 0} \ln(Tr(e^{-S_{class}}))$$
dove la forza di intrazione tra gli spin di fette temporali consecutivi è
$$K_{i}' = - \frac{1}{2} ln( tanh( \Delta \tau \Gamma_i) )$$
mentre la forza di interazione degli spin sulla stessa fetta temporale è
$$K_{ij} = \Delta \tau J_{ij}$$
Il numero di fette temporali $L_{\tau}$ è legato a $\Delta \tau = \frac{1}{T \cdot L_{\tau}}$ in modo che $\Delta \tau \to 0$ implica $L_{\tau} \to \infty$.
Per ottenere un algoritmo continuo ci interessa quindi avere un numero infinito di fette che equivale a dire $\Delta \tau \to 0$, ma essendo computazionalmente improponibile viene scelto di tenere $L_{\tau} \coloneqq M$ limitato considerando spin paralleli e adiacenti in intervalli temporali differenti, come \textbf{segmenti} temporali limitati agli estremi dai \textbf{cut} che li dividono da altri segmenti aventi lo spin nell'altro verso.

Prima di vedere l'algoritmo introduciamo le probabilità neccessarie al suo funzionamento iniziando dalle probabilità di due spin adiacenti di avere lo stesso verso a seconda se abbiano un legame temporale o spaziale:
la probabilità di due spin spazio-adiacenti di avere lo stesso verso è
$$p_{ij} = 1 - e^{-2K_{ij}} = 2\Delta \tau J_{ij} + O(\Delta \tau^2)$$
mentre la probabilità di due spin tempo-adiacenti di avere lo stesso verso è
$$p_i' = 1 - e^{-2K_{i}'} = 1 - \Delta \Gamma_{i} + O(\Delta \tau^2)$$
Queste probabilità vengono utilizzate attraverso l'algoritmo Monte Carlo per introdurre altri cut.
Da qui si può esprimere la probabilità di un certo spin $i$ di collegarsi al suo tempo-adiacente in un segmento di lunghezza $t < \beta$:
$$\hat{p_i} = \lim_{\Delta \tau \to 0} p_i'^{\frac{t}{\Delta \tau}} = \lim_{\Delta \tau \to 0} (1 - \Delta \tau \Gamma_i)^\frac{t}{\Delta \tau} = e^{- \Gamma_i t}$$
che servirà all'algoritmo per introdurre nuovi cut con probabilità
$$p_{cut}(i) = 1 - \hat{p_i}$$
con probabilità che diminuisce col diminuire della lunghezza.
Con lo stesso ragionamento di può anche dedurre la probabilità di un certo spin $ij$ di non collegare due segmenti $S_i\{[t_1, t_2]\}$ e $S_j\{[t_3, t_4]\}$:
$$\hat{p_{ij}} = \lim_{\Delta \tau \to 0} (1 - p_{ij})^\frac{t}{\Delta \tau} = \lim_{\Delta \tau \to 0} (1 - 2\Delta \tau J_{ij})^\frac{t}{\Delta \tau} = e^{-2 t J_{ij}}$$
con $t$ la lunghezza dell'intervallo $[t_1, t_2] \cap [t_3, t_4]$; che servirà all'algoritmo per collegare due segmenti creando un \textbf{cluster} con probabilità
$$p_{join}(i,j) = 1 - \hat{p_{i,j}}$$

L'algoritmo inizia creando come nel DT-SQA un modello TIM a partire da un modello IM inizializzato in maniera casuale. Per ogni spin si esegue un tentativo di taglio con probabilità $p_{cut}(i)$. Successivamente per ogni spin si esegue un tentativo di unione a cluster con probabilità $p_{join}(i,j)$. Il valore di spin di ogni cluster verrà invertito con probabilità $p = 0.5$, successivamente verrà calcolata l'energia del sistema.

% algoritmo
% inizializza tutto il TIM in un verso
% vengono generati M tagli
% vengono creati i join
% viene generato un assegnamento casuale
% viene calcolata l'energia
% vengono rimossi i tagli in eccesso e rimossi i join
% si ripete per N volte
