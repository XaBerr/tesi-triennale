Si prende il modello TIM n-dimensionale e lo si trasforma in un modello IM (n+1)-dimensionale attraverso l'approssimazione Suzuki-Trotter con condizioni al contorno periodiche. Otteniamo la solita funzione classica di energia:
$$S_{class} = - \sum_{\tau,\langle ij \rangle} K_{ij} S_i(\tau) S_j(\tau) - \sum_{\tau, i} K_{ij}' S_i(\tau) S_i(\tau+1)$$
e la utilizziamo per esprimere l'energia libera del sistema:
$$F = - T \cdot \lim_{\Delta \tau \to 0} \ln(Tr(e^{-S_{class}}))$$
dove la forza di intrazione tra gli spin dei tagli temporali consecutivi è $K_{i}' = - \frac{1}{2} ln( tanh( \Delta \tau \Gamma_i) )$ mentre la forza di interazione degli spin sullo stesso taglio temporale è $K_{ij} = \Delta \tau J_{ij}$. Il numero dei tagli temporali $L_{\tau}$ è legato a $\Delta \tau = \frac{1}{T \cdot L_{\tau}}$ in modo che $\Delta \tau \to 0$ implica $L_{\tau} \to \infty$.

Per ottenere un algoritmo continuo ci interessa quindi avere $\Delta \tau \to 0$ ma essendo computazionalmente improponibile viene scelto di tenere $L_{\tau}$ limitato considerando spin paralleli e adiacenti in intervalli temporali differenti, come dei segmenti temporali limitati agli estremi da dei \textit{tagli} che li dividono dagli altri segmenti aventi lo spin nell'altro verso.
È possibile determinare la probabilità di due spin adiacenti di avere lo stesso verso:
la probabilità di due spin spazio-adiacenti di avere lo stesso verso è
$$p_{ij} = 1 - e^{-2K_{ij}} = 2\Delta \tau J_{ij} + O(\Delta \tau^2)$$
mentre la probabilità di due spin tempo-adiacenti di avere lo stesso verso è
$$p_i' = 1 - e^{-2K_{i}'} = 1 - \Delta \Gamma_{i} + O(\Delta \tau^2)$$
Queste probabilità vengono utilizzate attraverso l'algoritmo Monte Carlo per introdurre altri tagli attraverso lo schema di aggiornamento non locale di claster Swendsen-Wang.
