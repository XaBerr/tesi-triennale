\lvlii{Nozioni}
\lvliii{Josephson Junction}
Josephson Junction è una giunzione tra due superconduttori, che hanno un accoppiamento debole, uniti attraverso un sottile strato isolate che svolge il ruolo di barriera di potenziale. Ci sono vari modi per realizzare una giunzione Josephson, abbiamo due categorie principali per i metalli distinte in base alle temperature utilizzate: per le temperature basse \idx{LTS} (Low Temperature Superconductors) oppure per gli ossidi a temperature elevate \idx{HTS} (High Temperature Superconductors). Per l'elettronica LTS le giunzioni ad effetto tunnel svolgono un ruolo molto importante grazie alla loro bassa dissipazione di energia. In particolare sono delle buone candidate per la costruzione dei qubit per le loro proprietà non lineari.

\lvliii{SQUID}
SQUID è un acronimo che sta per \textit{Superconducting Quantum Interference Devices} ovver per dispositivo ad interferenza quantistica superconduttivo. Sono dispositivi che che servono a misurare il flusso magnetico ed per indicarne il valore madano in output un voltaggio. Lo SQUID è un anello superconduttivo con una o più giunzioni Josephson.

\lvliii{Modello di Ising}

\lvliii{Problemi NP-completi}
